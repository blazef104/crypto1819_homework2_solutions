\chapter*{Exercise 2}

\section{point a}
\subsection{ i}%

\[
    \H \text{ is CRHF } \Rightarrow \H \text{ is OWF}
\]
To show this property, let's make a reduction:

\begin{figure}[h!]
   \centering
   \sdinit{}
   \begin{tikzpicture}
      \sdbegin{}
      \newinst{A}{$ \A'^{OWF} $}
      \newinst[4]{B}{$ \A $}
      \newinst[4]{C}{$\C^{CRHF}$}

      \postlevel
      \mess{C}{$s$}{B}
      \node[anchor=west] at (mess from) {\shortstack[l]{
      		$  s \leftarrow\$ \{0,1\}^{\lambda} 
      		  $ 
            \\
            $    $ }};
      \postlevel
      \mess{B}{$s, y=H_{s}(x)$}{A}
      \node[anchor=west] at (mess from) {\shortstack[l]{
      		$  x \leftarrow\$ \{0,1\}^{2n} 
      		  $ 
             }};
      \postlevel
      \mess{A}{$x'$}{B}
      \node[anchor=west] at (mess to) {  };
      \postlevel
      \mess{B}{$x, x'$}{C}
      \node[anchor=west] at (mess to) {  };


      \sdend{}
      \sdend{}
   \end{tikzpicture}
\end{figure}

When does not $\A$ win?\\
Since CRHF game  wants the final couple $(x, x')$ with $x\not= x'$, if
$\A'^{OWF}$ returns $x'=x$ the CRHF game doesn't work.\\

This \textbf{BAD} event happens with 
\[
    \P [ x=x' ]=Col(X, X')=\sum_{x}^{} \P [ X=x \wedge X'=x ]= \sum_{x}^{} \P [
    X=x] \P [ X'=x ] = \frac{1}{2^{2n}}        
\].
\subsection{ ii}%

If functions from $\H$ family aren't compressing, the probability of
\textbf{BAD} event changes:
\[
    \P [ x=x' ]=Col(X, X')=\sum_{x}^{} \P [ X=x \wedge X'=x ]= \sum_{x}^{} \P [
    X=x] \P [ X'=x ] = \frac{1}{2^{n}}        
\].

Now, if our functions from $\H$ were compressing (from 2n bits to n bits), the
best CRHF function (the function with the minimum number of collisions) had
$2^{n} + 1 $ inputs generating a collision (in the same codomain's element).\\

In this case, the best possible CRHF function is bijective (since it could be a
permutation over $2^{n}$ elements).\\

In general, for non-compressing functions we can show that 
\[
    \H \text{ is CRHF mapping n bits to n bits}  \Rightarrow \H \text{ is OWF}
\]

with the same reduction of the above \textbf{point i} .
