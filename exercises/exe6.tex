\mychapter{6}{Exercise 6}
\section{point a}
Active $\implies$ Passive

\begin{figure}[h!]
    \centering
    \sdinit{}
    \begin{tikzpicture}
       \sdbegin{}
       \newinst{A}{$ \A^{P-ID} $}
       \newinst[4]{B}{$ \A^{A-ID} $}
       \newinst[4]{C}{$\C^{A-ID}$}

       \postlevel
       \mess{C}{$P_k$}{B}
       \mess{B}{$P_k$}{A}
       \postlevel

       \mess{A}{"Transcript"}{B}
       \mess{C}{$\alpha_i$}{B}
       \mess{B}{$\beta_i$}{C}
       \mess{C}{$\gamma_i$}{B}
       \mess{B}{$\tau_i$}{A}
       \draw [->] (1.1,-6.3) to[out=240,in=120] (1.2,-3.3);
       \postlevel
       
       \mess{A}{$\tau^*$}{B}
       \mess{B}{$\tau^*$}{C}
       \node[anchor=west] at (mess to) { \shortstack{Output 1 IFF \\
       $V(P_k,\tau^*)=1$} };
 
       \sdend{}
       \sdend{}
    \end{tikzpicture}
 \end{figure}

We can construct a $\Pi_{BAD}$ s.t. Passive $\not\Rightarrow$ Active, where if we play as a dishonest Verifier we can leak the entire $S_k$.

For the passive Game:

\begin{figure}[h!]
    \centering
    \sdinit{}
    \begin{tikzpicture}[scale=0.8]
       \sdbegin{}
       \newinst{A}{$ \A^{P-ID} $}
       \newinst[4]{B}{$ \C^{P-ID} $}

       \postlevel
       \mess{B}{$P_k$}{A}
       \postlevel

       \mess{A}{"Transcript"}{B}
       \mess{B}{$\tau_i$}{A}
       \draw [->] (1.1,-4) to[out=240,in=120] (1.2,-3.0);
       \postlevel
       
       \mess{A}{$\tau^*$}{B}
       \node[anchor=west] at (mess to) { \shortstack{Output 1 IFF \\
       $V(P_k,\tau^*)=1$} };
 
       \sdend{}
       \sdend{}
    \end{tikzpicture}
\end{figure}

\newpage

Now $\tau_i$ will be as follows, from the point of view of $A$:

\begin{itemize}
    \item $\alpha\leftarrow U$
    \item $\beta=b||0$ where $b\leftarrow\$ U$
    \item $\gamma=\beta[-1](S_k)+(1-\beta[-1])(\alpha \beta)^{S_k}$
\end{itemize}

So the above ID-scheme has still Passive Security. Instead for the Active Security we can play the following interaction:

\begin{figure}[h!]
    \centering
    \sdinit{}
    \begin{tikzpicture}
       \sdbegin{}
       \newinst{A}{$ \A^{P-ID} $}
       \newinst[4]{B}{$ \C^{P-ID} $}

       \postlevel
       \mess{B}{$P_k$}{A}
       \postlevel

       \mess{B}{$\alpha$}{A}
       \mess{A}{$\beta$}{B}
       \node[anchor=west] at (mess to) { $\beta=b||1$ };
       \postlevel
       \mess{B}{$\gamma$}{A}
       \node[anchor=west] at (mess from) { \shortstack[l]{If $\beta[-1]=0,  \gamma=(\alpha \beta)^{S_k}$\\otw $\gamma=S_k$
       } };
       \draw [->] (1.1,-5) to[out=240,in=120] (1.2,-3);
       \postlevel
       
       \mess{A}{$\tau^*$}{B}
       \node[anchor=west] at (mess to) { \shortstack{Output 1 IFF \\
       $V(P_k,\tau^*)=1$} };
 
       \sdend{}
       \sdend{}
    \end{tikzpicture}
\end{figure}

Now that we have $S_k$ we can always create a valid $\tau^*$.

\section{point b}

\begin{figure}[h!]
    \centering
    \sdinit{}
    \begin{tikzpicture}
       \sdbegin{}
       \newinst{A}{$ \A^{A-ID} $}
       \newinst[4]{B}{$ \A^{UF-CMA} $}
       \newinst[4]{C}{$\C^{UF-CMA}$}

       \postlevel
       \mess{C}{$P_k$}{B}
       \mess{B}{$P_k$}{A}
       \postlevel

       \mess{A}{$m_i$}{B}
       \mess{B}{$m_i$}{C}
       \mess{C}{$\sigma_i$}{B}
       \mess{B}{$\sigma_i$}{A}
       \draw [->] (1.1,-6) to[out=240,in=120] (1.2,-3.3);
       \postlevel
       
       \mess{A}{$\tau^*$}{B}
       \node[anchor=west] at (mess to) { parse $\tau^*$ };
       \mess{B}{$m^*,\sigma^*$}{C}
       \node[anchor=west] at (mess to) { \shortstack{Output 1 IFF \\
       $Vrf(P_k,\sigma^*,m^*)=1$} };
 
       \sdend{}
       \sdend{}
    \end{tikzpicture}
\end{figure}

\newpage
\section{point c}

The HVZK property comes from the fact that the signature scheme used in the following way:

\begin{itemize}
    \item The Verifier sends the message
    \item The Prover signs the message
\end{itemize} 

doesn't leak any information about the secret used for signing the message (assuming $A$ always as honest verifier). 