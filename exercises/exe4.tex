\mychapter{4}{Exercise 4}

\section{point a}

Formal definition of CCA1. Consider the following $GAME^{CCA}$

\begin{figure}[h!]
    \centering
    \sdinit{}
    \begin{tikzpicture}
       \sdbegin{}
       \postlevel
       \newinst{A}{$ \A_{(P_{kb})} $}
       \newinst[4]{B}{$\C^{CCA1}_{(P_{kb},S_{kb})}$}
       \postlevel
       
       \mess{A}{$c$}{B}
       \node[anchor=east] at (mess from) { $Enc(P_k,m)=c$ };
       \mess{B}{m}{A}
       \draw [->] (1.1,-3.3) to[out=220,in=110] (1.2,-2.3);
       \postlevel
       
       \mess{A}{$m^*_0,m^*_1$}{B}
       \node[anchor=west] at (mess to) { $b\leftarrow\$ \{0,1\}$ };
       \mess{B}{$c^*$}{A}
       \postlevel

       \mess{A}{$b'$}{B}
       \node[anchor=west] at (mess to) { output 1 IFF $b=b'$ };

       \sdend{}
       \sdend{}
    \end{tikzpicture}
\end{figure}

$$|Pr[A(\lambda, 0)=1]-Pr[A(\lambda,1)=1]|\leq negl(\lambda)$$

\newpage
\section{point b}

CCA1 $\implies$ CPA

Assume $\exists \A^{CPA}$ which is able to break $CPA$. $\A^{CCA1}$ will use this $\A^{CPA}$ to break $CCA1$  

\begin{figure}[h!]
    \centering
    \sdinit{}
    \begin{tikzpicture}
       \sdbegin{}
       \postlevel
       \newinst{A}{$ \A^{CPA} $}
       \newinst[4]{B}{$\A^{CCA1}$}
       \newinst[4]{C}{$\C^{CCA1}$}
       \postlevel
       
       \mess{B}{$c$}{C}
       \mess{C}{m}{B}
       \draw [->] (6.8,-3.3) to[out=220,in=110] (6.8,-2.3);
       \postlevel
       
       \mess{A}{$m^*_0,m^*_1$}{B}
       \node[anchor=west] at (mess to) { };
       \mess{B}{$c^*$}{A}
       \postlevel

       \mess{B}{$m^*_0,m^*_1$}{C}
       \node[anchor=west] at (mess to) { };
       \mess{C}{$c^*$}{B}
       \postlevel

       \mess{A}{$b'$}{B}
       \mess{B}{$b'$}{C}
       \node[anchor=west] at (mess to) { output 1 IFF $b=b'$ };

       \sdend{}
       \sdend{}
    \end{tikzpicture}
\end{figure}

\noindent Intuitively this works because we used CPA to define CCA security. Therefore if an attacker is able to break CPA he is also "automatically" able to break CCA (the challenge part for CPA is the same for CCA).

\bigskip
$PKE^{CPA} \implies CCA1$

Now consider the following Game which is still CPA secure but on the other hand it leaks the key whenever $C$ receives a decryption query.

The scheme is defined as follows
\begin{figure}[h!]
    \centering
    \sdinit{}
    \begin{tikzpicture}
       \sdbegin{}
       \postlevel
       \newinst{A}{$ \A_{(P_{kb})} $}
       \newinst[4]{B}{$\C^{CCA1}_{(P_{kb},S_{kb})}$}
       \postlevel
       
       \mess{A}{$c_i$}{B}
       \node[anchor=east] at (mess from) { $Enc(P_k,m)=c$ };
       \mess{B}{$m_i,S_k$}{A}
       \node[anchor=west] at (mess from) { $Dec(S_k,c)=m,S_k$ };
       \draw [->] (1.1,-3.3) to[out=220,in=110] (1.2,-2.3);
       \postlevel
       
       \mess{A}{$m^*_0,m^*_1$}{B}
       \node[anchor=west] at (mess to) { $b\leftarrow\$ \{0,1\}$ };
       \mess{B}{$c^*$}{A}
       \node[anchor=east] at (mess to) { $Dec(x,c)=m_{b},x$ };
       \postlevel

       \mess{A}{$b'$}{B}
       \node[anchor=west] at (mess to) { output 1 IFF $b=b'$ };
       \node[anchor=east] at (mess from) { \shortstack{check if $m'=m_1$\\ or $m'=m_0$} };

       \sdend{}
       \sdend{}
    \end{tikzpicture}
\end{figure}

\textbf{Correctness:}

$Dec(S_k,\underbrace{Enc(P_k,m)}_{c})=m,S_k$

The scheme is still correct since in the decryption query I will still have the message as output + a second member which is the leaked key. 

\newpage
\section{point c}
\subsection{point i}

My goal is to demonstrate if $\Pi$ is $CCA1 \Rightarrow \Pi'$ is also CCA1, in order to do this observe the following reduction scheme:

Suppose $\exists A^{\Pi'}$ which is able to break the CCA1 security of $\Pi'$
\begin{figure}[h!]
    \centering
    \sdinit{}
    \begin{tikzpicture}
       \sdbegin{}
       \postlevel
       \newinst{A}{$ \A^{\Pi'} $}
       \newinst[4]{B}{$\A$}
       \newinst[4]{C}{$C$}
       \postlevel
       
       \mess{A}{$c[t]$}{B}
       \mess{B}{$c_i$}{C}
       \mess{C}{$m_i$}{B}
       \draw [->] (6.8,-3.8) to[out=220,in=110] (6.8,-2.9);
       
       \mess{B}{$c[t]$}{A}
       \postlevel

       \mess{A}{$m^*_0,m^*_1$}{B}
       \node[anchor=west] at (mess to) { fix $b=0$ };
       \mess{B}{$m^*_0[1,..,t-1]$}{C}
       \postlevel
       \mess{B}{$m^*_0[t],1$}{C}
       \node[anchor=west] at (mess to) { $b^c\leftarrow\$\{0,1\}$ };
       \mess{C}{$c^{*\prime}$}{B}
       
       \mess{B}{$c^*$}{A}       
       \node[anchor=west] at (mess from) { $c^*=c^*[1,..,t]||c^{*\prime}$ };
       
       \postlevel

       \mess{A}{$b'$}{B}
       \mess{B}{$b''$}{C}
       \node[anchor=west] at (mess to) { output 1 IFF $b^c=b''$ };

       \sdend{}
       \sdend{}
    \end{tikzpicture}
\end{figure}


$A^{\Pi'}$ sends ciphertext composed by $t$ elements. A takes every single elements and sends to C to get the plaintext of each one.
Then recombines the plaintext and sends back the single plaintext to $A^{\Pi'}$.

At the start of the challenge, $A^{\Pi'}$ sends two messages: $m_0$, $m_1$ of $t$ bits to $A$.
$A$ sends to $C$ the first t-1 bytes of $m_0$ and receives the corresponding $t-1$ ciphertexts then $A$ sends to $C$ the challenge as: $m_0[t]$ and 1, receiving the ciphertext $c*'$ of one of the two.
At this point A recombines all of the t-1 ciphertext + the last received, $c^{*\prime}$ and sends back to $A^{\Pi}$.
Now A will just forward the response.

The probability will be $|P[A^{\Pi}=0 | b^c=0]-P[A^{\Pi}=0 | b^c=1]|=\frac{1}{2}+negl(\lambda)-\frac{1}{2}>negl(\lambda)$ 

\newpage
\subsection{point ii}

$\Pi$ CCA2 $\implies$ $\Pi' \neg$CCA2

Consider the following PKE Scheme:
\begin{itemize}
    \item $Enc(P_k,m[t])=Enc(P_k,m_1)||...||Enc(P_k,m_t)$
    \item $Dec(S_k,c[t])=Dec(S_k,c_1)||...||Dec(S_k,c_t)$
\end{itemize}
Since in CCA2 I can make decryption queries after the challenge, I can create a $c'\neq c^*$ just by inverting the first two bits of $c^*$ ($c*=c_1^*||c_2^*||...||c^*_t$ now $c'=c_2^*||c_1^*||...||c^*_t$). Now when I receive the decrypted message I can simply switch the first two bits again and discover which of the two challenge messages was encrypted.

\section{point d}

From the definition of padded-RSA I can construct the following attack:

Suppose we do the challenge query, when I receive $C^*$ I will have something in this form: $c^*=(r||m_b)^e mod N$. Now, since RSA is malleable, I can change $C^*$ in order to be able to ask a valid decryption query, therefore $C'=C^*\times(r')^e mod N= ((r||m_b) \times r')^e\neq C^*$. Now when I ask for the decryption of $c'$ I will get $m'=((r||m)r')^{ed}=(r||m)r'$ since I know $r'$ I can simply divide $\frac{m'}{r'}$ and take the last l bits. This was the encrypted message in the challenge. 